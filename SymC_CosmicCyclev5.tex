\documentclass[11pt]{article}

\usepackage[utf8]{inputenc}
\usepackage[margin=0.75in,top=0.5in]{geometry}
\usepackage{amsmath,amssymb,amsfonts}
\usepackage{hyperref}
\usepackage{graphicx}
\usepackage{array}
\usepackage{enumitem}
\setlist[itemize]{noitemsep,topsep=2pt}
\setlist[enumerate]{noitemsep,topsep=2pt}

% Define simple table rules if booktabs not available
\newcommand{\toprule}{\hline}
\newcommand{\midrule}{\hline}
\newcommand{\bottomrule}{\hline}

\title{\textbf{The Primordial Boundary Principle: \\Identifying Cosmic Acceleration with Exceptional Point Coalescence}}

\author{Nate Christensen\\
Independent Researcher\\
SymC Research Project, MO, USA\\
\texttt{NateChristensen@SymCUniverse.com}}

\date{04 February 2026}

\begin{document}
\maketitle

\begin{abstract}
Standard cosmological models attribute cosmic acceleration to dark energy's negative pressure. This work reinterprets the transition as a stability boundary crossing in structure formation dynamics. The linear growth equation for matter perturbations maps to a damped harmonic oscillator with damping ratio $\chi_{\delta} \equiv H/\sqrt{4\pi G \rho_m}$. In flat $\Lambda$CDM, $\chi_{\delta}=1$ (critical damping) coincides exactly with the transition from deceleration to acceleration ($q=0$), both occurring at $\Omega_m = 2/3$ ($z\approx 0.67$). This transition represents an Exceptional Point coalescence in the growth propagator, where poles merge and the system shifts from oscillatory to monotonic relaxation. The coincidence problem is reframed as a control-theoretic necessity: the universe maximizes structure formation efficiency near $\chi \approx 1$ before entering terminal overdamping. Falsifiable predictions include redshift alignment of structure freeze-out, scale-dependent growth modifications, and constraints on $w(a)$.
\end{abstract}

\section*{Note on terminology}

Throughout this work, \textit{critical} refers to the dynamical separatrix $\chi = 1$ for a second-order linear operator. The claim is not that the universe optimizes. The claim is that the dynamical operator contains a boundary separating oscillatory response from monotone response, and that long-lived trajectories relevant to structure will be concentrated near that boundary. This critical boundary (SymC) is used here as an organizing principle for stability and failure across scales, rather than as an additional dynamical assumption.

\section{Introduction}

The onset of cosmic acceleration at redshift $z \approx 0.7$ remains one of the most significant puzzles in modern cosmology \cite{Riess1998,Perlmutter1999}. In the standard $\Lambda$CDM model, this phenomenon is driven by a cosmological constant $\Lambda$ or a dark energy fluid with equation of state $w \approx -1$. While observationally robust, the coincidence of the vacuum energy density $\rho_{\Lambda}$ being comparable to the matter density $\rho_m$ at the present epoch presents a fine-tuning problem \cite{Weinberg1989}.

This work proposes that cosmic acceleration is not merely a kinematic feature of the background expansion but a dynamical stability transition in the growth of structure. The critical damping ratio $\chi = \gamma/(2|\omega|) = 1$ emerges as a universal boundary for field stability and information efficiency when applied to cosmological perturbations. 

\subsection{Cosmic acceleration as growth suppression}

Cosmic acceleration is usually described as a new push that dominates late-time expansion. The same phenomenon can be described, without changing the background model, as a late-time suppression of structure growth produced by the friction term in the linear growth equation. In this view, the dark sector is not just a background energy density but a control channel regulating the amplitude and timescale of matter clustering.

The central questions become: when does the growth operator cross the critical damping boundary, and how is that crossing related to the onset of background acceleration? Here, "primordial" refers to a foundational dynamical boundary constraining system trajectories, not to an early-time cosmological epoch.

\subsection{Two layers of the argument}

This manuscript operates at two levels:
\begin{enumerate}
\item A model-internal identity, derived within flat $\Lambda$CDM, that maps the onset of acceleration ($q=0$) to a critical damping condition for growth ($\chi_\delta = 1$).
\item An effective-medium interpretation in which dark matter and dark energy are unified as an elastic-viscous medium constrained by primordial boundary conditions.
\end{enumerate}

The first layer is the empirical spine: a clean, falsifiable statement about the relationship between background expansion and growth dynamics. The second layer is an effective-medium narrative about the dark sector and its inheritance from primordial boundary conditions \cite{ChristensenClosedGaps}. The two are logically separated: the identity does not depend on any particular substrate, and the medium-level interpretation is explicitly labeled as hypothesis where it goes beyond the identity.

\section{Growth dynamics as a damped oscillator}

\subsection{Linear growth equation}

For subhorizon, pressureless matter perturbations in standard cosmology, the linear growth factor $\delta$ obeys \cite{Peebles1980,Mukhanov2005}
\begin{equation}
\ddot{\delta} + 2H\dot{\delta} - 4\pi G \rho_m \delta = 0,
\label{eq:growth}
\end{equation}
where overdot denotes derivative with respect to cosmic time, $H = \dot{a}/a$ is the Hubble rate, and $\rho_m$ is the matter density.

\subsection{Oscillator mapping}

The canonical damped oscillator equation is
\begin{equation}
\ddot{x} + \gamma \dot{x} + \omega^2 x = 0,
\label{eq:osc}
\end{equation}
with damping coefficient $\gamma$ and squared frequency $\omega^2$. Mapping Eq. \eqref{eq:growth} to Eq. \eqref{eq:osc} yields
\begin{equation}
\gamma \equiv 2H, \qquad \omega^2 \equiv -4\pi G \rho_m.
\label{eq:map}
\end{equation}

Because $\omega^2 < 0$ in the growth equation, the restoring term corresponds to an unstable direction: gravity amplifies perturbations rather than pulling them back to equilibrium. Nevertheless, the ratio between friction and the characteristic rate still controls whether the response is oscillatory in phase space or monotone in the relevant variable.

\begin{table}[h]
\centering
\caption{Mapping between cosmological and oscillator variables}
\begin{tabular}{lll}
\toprule
\textbf{Oscillator Term} & \textbf{Cosmological Variable} & \textbf{Physical Meaning} \\
\midrule
Damping coefficient $\gamma$ & $2H$ & Hubble friction \\
Stiffness $\omega^2$ & $4\pi G\rho_m$ & Gravitational restoring force \\
Damping ratio $\chi$ & $H/\sqrt{4\pi G\rho_m}$ & Stability parameter \\
\bottomrule
\end{tabular}
\label{tab:mapping}
\end{table}

\subsection{Damping ratio for growth}

Define the dimensionless growth damping ratio
\begin{equation}
\chi_\delta \equiv \frac{\gamma}{2|\omega|}
= \frac{2H}{2\sqrt{4\pi G\rho_m}}
= \frac{H}{\sqrt{4\pi G\rho_m}}.
\label{eq:chi_def}
\end{equation}

Using $\rho_m = 3H^2\Omega_m/(8\pi G)$ for a flat universe,
\begin{equation}
\chi_\delta = \sqrt{\frac{2}{3\Omega_m}}.
\label{eq:chi_omega}
\end{equation}

The critical boundary $\chi_\delta = 1$ therefore corresponds to
\begin{equation}
\Omega_m = \frac{2}{3}.
\label{eq:om_23}
\end{equation}

For $\chi_\delta < 1$, the system is in an adaptive regime for growth: friction is not yet strong enough to suppress the unstable direction, and structure can form efficiently. Once $\chi_\delta > 1$, the system enters an overdamped regime where growth is monotonically suppressed.

\section{The critical identity}

\subsection{Equivalence to $q=0$ in flat $\Lambda$CDM}

In flat $\Lambda$CDM, the deceleration parameter is
\begin{equation}
q \equiv -\frac{\ddot{a}a}{\dot{a}^2}
= \frac{1}{2}\Omega_m - \Omega_\Lambda,
\label{eq:q_def}
\end{equation}
with $\Omega_m + \Omega_\Lambda = 1$.

The condition $q=0$ implies $\Omega_\Lambda = \Omega_m/2$ and hence $\Omega_m = 2/3$. Comparing with Eq. \eqref{eq:om_23}, the exact identity is obtained:
\begin{equation}
\boxed{
\chi_\delta = 1 \quad \Longleftrightarrow \quad q = 0
\qquad \text{(flat $\Lambda$CDM)}.
}
\label{eq:identity}
\end{equation}

This is not a new parameterization. It is a restatement of the same background transition in the language of the growth operator.

\subsection{Redshift of the crossing}

Let $\Omega_{m0}$ be the present-day matter fraction. In flat $\Lambda$CDM,
\begin{equation}
\Omega_m(z) = \frac{\Omega_{m0}(1+z)^3}{\Omega_{m0}(1+z)^3 + (1-\Omega_{m0})}.
\label{eq:omz}
\end{equation}

Setting $\Omega_m(z_\chi) = 2/3$ yields
\begin{equation}
1+z_\chi = \left(\frac{2(1-\Omega_{m0})}{\Omega_{m0}}\right)^{1/3}.
\label{eq:zchi}
\end{equation}

For $\Omega_{m0}\simeq 0.315$ (Planck 2018 \cite{PlanckCollaboration2020}), $z_\chi \simeq 0.67$, consistent with the standard transition redshift.

\section{Exceptional point spectral analysis}

\subsection{Linear operator structure}

The linear operator corresponding to Eq. \eqref{eq:growth} is
\begin{equation}
\mathcal{L} = \frac{d^2}{dt^2} + 2H\frac{d}{dt} - 4\pi G\rho_m.
\end{equation}

Seeking solutions of the form $\delta \propto e^{\lambda t}$, the characteristic equation is
\begin{equation}
\lambda^2 + 2H\lambda - 4\pi G\rho_m = 0,
\end{equation}
with eigenvalues
\begin{equation}
\lambda_{\pm} = -H \pm \sqrt{H^2 + 4\pi G\rho_m}.
\end{equation}

\subsection{Rewriting in terms of $\chi_{\delta}$}

Using $4\pi G\rho_m = H^2/\chi_{\delta}^2$,
\begin{equation}
\lambda_{\pm} = -H \pm H\sqrt{1 + \chi_{\delta}^{-2}} = -H\left(1 \mp \sqrt{1 + \chi_{\delta}^{-2}}\right).
\end{equation}

For the correct sign convention with the growth equation where $\omega^2 < 0$, rewrite as
\begin{equation}
\lambda_{\pm} = -H \pm \sqrt{H^2 - 4\pi G \rho_m} = -H\left(1 \mp \sqrt{1 - \chi_{\delta}^{-2}}\right),
\end{equation}
where the instability corresponds to $\lambda_+ > 0$ for growing modes when $\chi_\delta < 1$.

\subsection{Exceptional point coalescence}

At $\chi_{\delta}=1$, the eigenvalues coalesce \cite{Heiss2012,Kato1966,Berry2004,Rotter2009}:
\begin{equation}
\lambda_{+} = \lambda_{-} = -H.
\end{equation}

This eigenvalue degeneracy defines a second-order Exceptional Point (EP). The Jordan block becomes nondiagonalizable, and the propagator exhibits algebraic rather than exponential time dependence:
\begin{equation}
G(t) \sim (1 + Ht)e^{-Ht} \quad \text{(critical growth or decay)}.
\end{equation}

\begin{itemize}
\item For $\chi_{\delta} < 1$ (Matter Era): Poles are complex, corresponding to oscillatory modes capable of efficient clustering.
\item For $\chi_{\delta} = 1$ (Transition, $z \approx 0.7$): The poles coalesce at $\lambda = -H$. The discriminant vanishes. This is a second-order EP.
\item For $\chi_{\delta} > 1$ (Dark Energy Era): Poles split along the real axis. The system becomes overdamped.
\end{itemize}

The bifurcation of the propagator poles represents the universe losing the complex eigenvalues required for complex structure formation.

\section{Acceleration versus growth damping}

The vacuum component appears differently depending on which equation is considered. In the Friedmann acceleration equation,
\begin{equation}
\frac{\ddot{a}}{a} = -\frac{4\pi G}{3}(\rho + 3p/c^2),
\label{eq:fried_acc}
\end{equation}
a component with $p = -\rho c^2$ contributes negative active gravitational mass, so once it dominates, $\ddot{a}>0$.

In the growth equation \eqref{eq:growth}, the same component contributes primarily through $H(z)$, which multiplies the friction term $2H\dot{\delta}$. Thus, even while the scale factor accelerates, the growth of structure is damped. This operator-level distinction means the late-time universe is accelerating in $a(t)$ while braking in $\delta(t)$.

\section{Physical interpretation: Vacuum modes as effective descriptions}

\textbf{Note: This section presents an interpretive framework, not standard cosmology.}

The cosmological vacuum can be modeled as a physical substrate whose properties emerge in different limiting cases \cite{ChristensenClosedGaps}:

\begin{itemize}
\item \textbf{Elastic response limit ($\chi \ll 1$)}: At high matter density, the substrate responds primarily through potential energy storage. This effective stiffness manifests observationally as enhanced gravitational clustering, phenomenologically described as dark matter.

\item \textbf{Viscous response limit ($\chi \gg 1$)}: At low matter density, the substrate responds through dissipative drag. This effective viscosity manifests as accelerated expansion, phenomenologically described as dark energy.

\item \textbf{Critical response ($\chi \approx 1$)}: The substrate balances elastic and viscous responses, enabling maximal information processing and structure formation efficiency \cite{ChristensenAIF}.
\end{itemize}

These are not fundamental entities but effective descriptions of substrate behavior in different regimes, analogous to how elasticity and viscosity describe material properties under different conditions.

\subsection{Frequency injection: The origin (hypothesis)}

This subsection is hypothesis. It is not required for the identity in Eq. \eqref{eq:identity}.

The pre-universe is modeled as a vacuum-dominated state with effectively infinite damping ($\rho_m \approx 0, \chi \to \infty$). To initiate a structural cycle, the system must undergo a drastic reduction in $\chi$. Reheating is identified as the mechanism of frequency injection. The decay of the inflaton field transfers energy into matter degrees of freedom ($\rho_m$), effectively injecting stiffness or frequency ($\omega$) into the system:
\begin{equation}
\chi_{init} = \frac{H}{\sqrt{\omega_{inj}^2}} \ll 1.
\end{equation}

This injection resets the stability clock, creating a highly underdamped ($\chi \ll 1$) environment that permits the rapid growth of perturbations (acoustic peaks) and the formation of the CMB.

\subsection{Terminal rigidity: Type B failure}

Unlike biological systems which use feedback to maintain $\chi \approx 1$, the cosmological vacuum lacks a mechanism to down-regulate the damping term. The Hubble friction $H$ approaches a constant ($\sqrt{\Lambda/3}$) while the restoring force $\rho_m$ dilutes as $a^{-3}$. The trajectory is therefore asymptotic:
\begin{equation}
\chi_{\delta}(a) \propto \frac{1}{\sqrt{\Omega_m(a)}} \propto a^{3/2} \to \infty.
\end{equation}

This represents a Type B (Overdamped) Control Failure. The universe does not explode; it freezes. The rigidity refers to the inability of matter perturbations $\delta$ to grow, despite (and because of) the rapid expansion of the metric $a$.

\section{Falsification and quantitative predictions}

\subsection{Redshift alignment}

The peak of cosmic star formation rate density ($z_{\text{SFR, peak}} \approx 1.9$) should correlate with $\chi_{\delta}(z) \approx 0.8$, while the steep decline ($z < 1$) should track $\chi_{\delta}(z)$ approaching and crossing 1.

\subsection{Growth function suppression}

Define the suppression factor \cite{Linder2005,Guzzo2008}
\begin{equation}
S(z) \equiv \frac{f\sigma_8(z)_{\text{observed}}}{f\sigma_8(z)_{\Lambda\text{CDM}}}.
\end{equation}

The framework predicts $S(z)$ should show a characteristic bend at $z_\chi$, with
\begin{equation}
\frac{d^2S}{dz^2}\bigg|_{z_\chi} \neq 0.
\end{equation}

\subsection{Scale-dependent modifications}

The modified growth parameter $\mu(k,z)$ from the parameterization
\begin{equation}
\ddot{\delta} + 2H\dot{\delta} - 4\pi G \rho_m \left[1+\mu(k,z)\right]\delta = 0
\label{eq:mu}
\end{equation}
should exhibit specific redshift evolution:
\begin{equation}
\mu(k,z) \approx \mu_0(k)\left[1 - \exp\left(-\frac{(z - z_\chi)^2}{2\sigma_z^2}\right)\right],
\end{equation}
with $\sigma_z \approx 0.3$ characterizing the transition width.

\subsection{The $\Delta a$ split}

Define $\Delta a \equiv a_{q=0} - a_{\chi=1}$. In standard $\Lambda$CDM ($w=-1$), $\Delta a = 0$. A measurement of $\Delta a \neq 0$ would falsify the strict identity and imply dynamic dark energy ($w \neq -1$) or modified gravity.

\subsection{No-rebound constraint}

In flat $\Lambda$CDM, $\Omega_m(z)$ decreases monotonically to zero as $z\to -1$ (future). Equation \eqref{eq:chi_omega} then implies $\chi_\delta \to \infty$. Therefore, absent a deviation from $w=-1$ or modified gravity, the trajectory cannot curve back toward $\chi_\delta \simeq 1$. The equation of state $w(a)$ must not exhibit behavior that allows $\rho_{DE}$ to decay faster than $\rho_m$ in the future.

\section{Relation to other frameworks}

\subsection{Quintessence and modified gravity}

This interpretation differs fundamentally from quintessence models \cite{Copeland2006}: while quintessence introduces a new dynamical scalar field, the present work identifies acceleration as an emergent property of the structure formation dynamics itself. The damping ratio $\chi_{\delta}$ serves as an order parameter rather than a field potential.

\subsection{Holographic and thermodynamic approaches}

Like holographic dark energy models, this framework emphasizes information-theoretic considerations near critical points. However, the critical boundary is derived directly from the linear perturbation operator rather than from entropy-area relations.

\subsection{Emergent gravity}

Similar to Verlinde's emergent gravity, gravitational phenomena are treated as emergent. However, this work provides a specific dynamical mechanism (critical damping transition) rather than entropic forces.

\section{Limitations}

\subsection{Model dependence}

The equivalence $\chi_\delta = 1 \Leftrightarrow q = 0$ is exact only in flat $\Lambda$CDM. If $w \neq -1$, if curvature is nonzero, or if growth is modified, the mapping shifts. This is a feature for testing, not a flaw: departures are precisely where observations can discriminate between competing pictures.

\subsection{Status of substrate identification}

Specific candidate identifications at the microphysical level remain constrained and in places challenged by available estimates. For example, a candidate discussed in prior work yields an inferred range $\chi \sim 0.06$ to $0.12$ under the simplest parameter mapping, far from critical. The cosmology identity result does not depend on choosing a specific substrate.

\section{Conclusion}

An exact identity has been demonstrated within flat $\Lambda$CDM cosmology: the critical damping condition for structure formation ($\chi_{\delta}=1$) coincides precisely with the kinematic transition to cosmic acceleration ($q=0$). This mathematical equivalence provides a new perspective on cosmic acceleration, reframing it as a stability boundary crossing rather than the action of an exogenous dark energy fluid.

This framework interprets the transition as an exceptional point coalescence in the growth propagator, where the universe shifts from an underdamped regime supporting structure formation to an overdamped regime of structural isolation. This naturally explains the why now coincidence: the universe is observed near its peak information-processing efficiency, just after crossing the critical boundary.

While the $\chi_{\delta}=1 \Leftrightarrow q=0$ identity is a theorem within $\Lambda$CDM, the broader framework provides falsifiable predictions:
\begin{itemize}
\item Precise alignment between structure formation freeze-out and $z_\chi$
\item Characteristic scale-dependent modifications to growth
\item The impossibility of rebound to $\chi<1$ unless $w\neq-1$
\item A measurable $\Delta a$ split for non-$\Lambda$CDM cosmologies
\end{itemize}

Future surveys (DESI, Euclid, Roman) will test these predictions with unprecedented precision, potentially validating or falsifying this interpretation of cosmic acceleration as a control system failure.

\section*{Acknowledgments}

The author is grateful to the broader cosmology, gravitation, and quantum field theory communities for the observational and theoretical groundwork on which this analysis builds. The availability of precise cosmological datasets and open theoretical resources has been central to the development of the results presented in this work.


\section*{References}

\begin{thebibliography}{99}

\bibitem{Riess1998}
Riess, A. G., Filippenko, A. V., Challis, P., et al. (1998).
Observational evidence from supernovae for an accelerating universe and a cosmological constant.
\textit{Astronomical Journal, 116}, 1009–1038.

\bibitem{Perlmutter1999}
Perlmutter, S., Aldering, G., Goldhaber, G., et al. (1999).
Measurements of $\Omega$ and $\Lambda$ from 42 high-redshift supernovae.
\textit{Astrophysical Journal, 517}, 565–586.

\bibitem{Weinberg1989}
Weinberg, S. (1989).
The cosmological constant problem.
\textit{Reviews of Modern Physics, 61}, 1–23.

\bibitem{Copeland2006}
Copeland, E. J., Sami, M., \& Tsujikawa, S. (2006).
Dynamics of dark energy.
\textit{International Journal of Modern Physics D, 15}, 1753–1936.

\bibitem{PlanckCollaboration2020}
Planck Collaboration. (2020).
Planck 2018 results. VI. Cosmological parameters.
\textit{Astronomy \& Astrophysics, 641}, A6.

\bibitem{Peebles1980}
Peebles, P. J. E. (1980).
\textit{The large-scale structure of the universe}.
Princeton University Press.

\bibitem{Mukhanov2005}
Mukhanov, V. (2005).
\textit{Physical foundations of cosmology}.
Cambridge University Press.

\bibitem{Lahav1991}
Lahav, O., Lilje, P. B., Primack, J. R., \& Rees, M. J. (1991).
Dynamical effects of the cosmological constant.
\textit{Monthly Notices of the Royal Astronomical Society, 251}, 128–136.

\bibitem{Linder2005}
Linder, E. V. (2005).
Cosmic growth history and expansion history.
\textit{Physical Review D, 72}, 043529.

\bibitem{Kaiser1987}
Kaiser, N. (1987).
Clustering in real space and in redshift space.
\textit{Monthly Notices of the Royal Astronomical Society, 227}, 1–21.

\bibitem{Hamilton1998}
Hamilton, A. J. S. (1998).
Linear redshift distortions.
\textit{Astrophysical Journal, 499}, 555–564.

\bibitem{Guzzo2008}
Guzzo, L., Pierleoni, M., Meneux, B., et al. (2008).
A test of the nature of cosmic acceleration using galaxy redshift distortions.
\textit{Nature, 451}, 541–545.

\bibitem{MaBertschinger1995}
Ma, C.-P., \& Bertschinger, E. (1995).
Cosmological perturbation theory in the synchronous and conformal Newtonian gauges.
\textit{Astrophysical Journal, 455}, 7–25.

\bibitem{Eckart1940}
Eckart, C. (1940).
The thermodynamics of irreversible processes. III.
\textit{Physical Review, 58}, 919–924.

\bibitem{IsraelStewart1979}
Israel, W., \& Stewart, J. M. (1979).
Transient relativistic thermodynamics.
\textit{Annals of Physics, 118}, 341–372.

\bibitem{RezZan2013}
Rezzolla, L., \& Zanotti, O. (2013).
\textit{Relativistic hydrodynamics}.
Oxford University Press.

\bibitem{Kato1966}
Kato, T. (1966).
\textit{Perturbation theory for linear operators}.
Springer.

\bibitem{Berry2004}
Berry, M. V. (2004).
Physics of non-Hermitian degeneracies.
\textit{Czechoslovak Journal of Physics, 54}, 1039–1047.

\bibitem{Rotter2009}
Rotter, I. (2009).
A non-Hermitian Hamilton operator and the physics of open quantum systems.
\textit{Journal of Physics A, 42}, 153001.

\bibitem{Heiss2012}
Heiss, W. D. (2012).
The physics of exceptional points.
\textit{Journal of Physics A: Mathematical and Theoretical, 45}, 444016.

\bibitem{Strogatz1994}
Strogatz, S. H. (1994).
\textit{Nonlinear dynamics and chaos}.
Westview Press.

\bibitem{Ashwin2014}
Ashwin, P., Wieczorek, S., Vitolo, R., \& Cox, P. (2014).
Tipping points in open systems.
\textit{Physical Review E, 90}, 012901.

\bibitem{ChristensenAIF}
Christensen, N. (2025).
The adaptive inference framework (AIF): The SymC boundary principle for information efficiency and critical thinking optimization.
Zenodo. https://doi.org/10.5281/zenodo.17904559

\bibitem{ChristensenBH}
Christensen, N. (2026).
Structural mapping of linear damping operators across cosmological growth and gravitational ringdown.
Zenodo. https://doi.org/10.5281/zenodo.18500138

\bibitem{ChristensenQFT}
Christensen, N. (2026).
Structural constraints from critical damping in open quantum field theories: Implications for QCD substrate inheritance and phenomenological extensions.
Zenodo. https://doi.org/10.5281/zenodo.18476360

\bibitem{ChristensenClosedGaps}
Christensen, N. (2026).
Closing critical gaps: Physical inheritance from stabilized substrates in dynamical systems.
Zenodo. https://doi.org/10.5281/zenodo.18452784

\end{thebibliography}

\end{document}