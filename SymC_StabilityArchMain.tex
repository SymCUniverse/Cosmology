\documentclass[12pt]{article}
\usepackage[top=0.5in, left=0.75in, right=0.75in, bottom=0.75in]{geometry}
\usepackage{amsmath,amssymb}
\usepackage{graphicx}
\usepackage{float}
\usepackage{cite}
\usepackage{url}
\usepackage[hidelinks]{hyperref}
\usepackage{titlesec}

\titleformat{\section}{\large\bfseries}{\thesection}{1em}{}
\titleformat{\subsection}{\normalsize\bfseries}{\thesubsection}{1em}{}

\title{\textbf{Exceptional-Point Lineage and the Stability Architecture of Physical Reality: Symmetrical Convergence Across Quantum and Cosmic Scales}}

\author{Nate Christensen\\
\small SymC Universe Project, Missouri, USA\\
\small \texttt{NateChristensen@SymCUniverse.com}}

\date{09 February 2026}

\begin{document}

\maketitle

\begin{abstract}
\noindent Physical reality is organized by structural boundaries. The dimensionless damping ratio $\chi \equiv \gamma/(2|\omega|)$ defines a stability architecture in which the critical threshold $\chi=1$ marks a second-order non-Hermitian Exceptional Point (EP). A parameter-free identity is derived linking cosmic acceleration to critical damping, $\chi_{\delta}=1 \iff q=0$, indicating a structural constraint on the timing of dark energy dominance. The result is structural rather than causal: it follows from the form of the growth and Friedmann equations within flat $\Lambda$CDM, independent of any microscopic model for dark energy.
This stability condition propagates through a substrate inheritance relation, whereby emergent modes inherit EP structure from vacuum precursors. Within this framework, the observed particle distribution is consistent with stability organization: long-lived matter occupies localized stability basins at $\chi \ll 1$, while short-lived excitations cluster near the $\chi=1$ interface. The framework replaces ad hoc parameter tuning with a stability-based classification principle.
\end{abstract}

% ============================================================
\section{Boundary Structure in Dissipative Dynamics}
% ============================================================

Physics is organized by boundaries. The speed of light $c$ limits information propagation, and Planck's constant $\hbar$ bounds measurement precision. The dimensionless ratio
\begin{equation}\label{eq:chi_def}
\chi \equiv \frac{\gamma}{2|\omega|}
\end{equation}
plays an analogous role in dynamical stability.

At $\chi = 1$, the generator of dynamics becomes defective and the impulse response transitions to
\begin{equation}\label{eq:ep_kernel}
h(t) = t e^{-|\omega|t},
\end{equation}
defining a second-order non-Hermitian exceptional point (EP2) \cite{heiss2012,kato1995,bender2007}.

Quadratic eigenproblems for linear response arise across classical dissipative modes, retarded propagators in open quantum field theory, and cosmological growth dynamics \cite{breuer2002,carmichael1999,peebles1993,dodelson2003}. In each case, the exceptional-point kernel appears when the damping rate matches twice the characteristic frequency, $\chi=\gamma/(2|\omega|)=1$. Markovian open-system generators of Lindblad type can realize the same EP2 structure in suitable reduced response channels or effective mode equations (Supplementary Section~1) \cite{lindblad1976,gorini1976}, but the EP condition is asserted here at the level of the shared quadratic response structure.

For representative linear response channels with Gaussian noise and finite bandwidth, an information-efficiency functional $\eta(\chi)\equiv I(\chi)/\Sigma(\chi)$ admits a strict local maximum in a narrow neighborhood of $\chi=1$ under standard smoothness conditions (Supplementary Section 2). This supports the interpretation of near-critical damping as a robust operating point balancing responsiveness against dissipation in common model classes.

A strict local maximum at $\chi=1$ follows if $\eta(\chi)$ is twice differentiable in a neighborhood of $\chi=1$, $\eta'(1)=0$, and $\eta''(1)<0$. The Gaussian-channel examples provide an explicit realization of these conditions.

Most significantly, the onset of cosmic acceleration ($q=0$) is mathematically identical to the structure growth field reaching critical damping ($\chi_\delta = 1$), providing a structural explanation for the timing of late-time acceleration without fine-tuning.

% ============================================================
\section{Exceptional-Point Kernel and Efficiency Extremum}
% ============================================================

For a representative linear mode whose observable response obeys a second-order dissipative equation:
\begin{equation}
\ddot{x} + \gamma \dot{x} + \omega^2 x = 0 ,
\end{equation}
the characteristic discriminant
\begin{equation}
\Delta = \gamma^2 - 4\omega^2 = 4\omega^2(\chi^2 - 1)
\end{equation}
vanishes at $\chi = 1$, yielding coalesced roots $\lambda_\pm = -|\omega|$ and the EP2 impulse kernel \eqref{eq:ep_kernel} \cite{kato1995}.
Such quadratic response structures appear directly in classical linear-response theory and in retarded propagators of dissipative quantum field theory \cite{breuer2002}. Lindblad-form Markovian dynamics can realize the same EP2 kernel in appropriate effective response channels; the mapping and its scope are given explicitly in Supplementary Section~1 \cite{lindblad1976,gorini1976}.

An identical structure arises in dissipative quantum field theory. For a scalar mode with retarded propagator
\begin{equation}
G_R(\Omega) = \frac{1}{-\Omega^2 - i\gamma \Omega + \omega^2},
\end{equation}
the poles
\begin{equation}
\Omega_\pm = -\frac{i\gamma}{2} \pm \sqrt{\omega^2 - \frac{\gamma^2}{4}}
\end{equation}
coalesce at $\gamma = 2|\omega|$, reproducing the same EP2 kernel \cite{breuer2002}.

The information-efficiency functional
\begin{equation}
\eta(\chi) \equiv \frac{I(\chi)}{\Sigma(\chi)}
\end{equation}
exhibits a strict local maximum at $\chi = 1$ in Gaussian channel models (Supplementary Section~2). This identifies $\chi=1$ as a recurrent structural boundary in systems that simultaneously admit responsiveness and stability.

Physically, $\chi<1$ corresponds to persistent ringing, while $\chi>1$ sacrifices responsiveness. Realistic constraints broaden the optimum to a narrow window near criticality \cite{ogata2010}. Figure~\ref{fig:topology} visualizes this dual structure as an exceptional-point manifold separating overdamped and underdamped regimes.

\begin{figure}[H]
\centering
\includegraphics[width=0.95\linewidth]{Fig1_VacMatterTopology.png}
\caption{\textbf{Topological structure of the stability manifold.} Three-dimensional visualization in $(\log_{10}\omega, \log_{10}\Gamma, \chi)$ space where $\chi = \Gamma/(2\omega)$ is the dimensionless damping ratio. The colored surface represents the exceptional point (EP) manifold at $\chi = 1$, separating overdamped ($\chi > 1$, magenta) and underdamped ($\chi < 1$, blue) dynamical regimes. Representative systems illustrate an ordering by observed stability index: vacuum-lineage processes appear near the critical line, while long-lived excitations lie deep in the underdamped basin. The schematic language is classificatory and does not assert a microscopic pathway. Vertical projection lines connect the 3D topology to the empirical validation plane (Figure~\ref{fig:lineage}).}
\label{fig:topology}
\end{figure}

\subsection*{Structural interpretation}
The exceptional-point boundary at $\chi=1$ is not interpreted as a dynamical attractor. Rather, it serves as a structural separatrix where impulse response transitions from oscillatory to monotone decay. In representative dissipative channels (second-order systems with Gaussian noise and finite bandwidth), the information-efficiency functional $\eta(\chi)$ admits a strict local maximum at $\chi=1$. Departures from $\chi\approx 1$ incur complementary penalties: $\chi<1$ produces persistent ringing that degrades signal fidelity under finite bandwidth, while $\chi>1$ produces slow relaxation that suppresses information throughput for fixed dissipation. The framework interprets near-critical operation as a robust efficiency extremum, consistent with observed system clustering in the range $0.8\lesssim\chi\lesssim1.0$.

% ============================================================
\section{Cosmic Crossing: $\chi_\delta = 1 \Longleftrightarrow q = 0$}
% ============================================================

Linear density perturbations satisfy
\begin{equation}
\ddot{\delta} + 2H\dot{\delta} - 4\pi G\rho_m \delta = 0
\end{equation}
\cite{peebles1993,dodelson2003}, yielding
\begin{equation}
\chi_\delta = \frac{H}{\sqrt{4\pi G\rho_m}} .
\end{equation}

In flat $\Lambda$CDM, the deceleration parameter
\begin{equation}
q = \frac{1}{2}\Omega_m - \Omega_\Lambda
\end{equation}
vanishes at $\Omega_m = 2/3$. The Friedmann equations then give
\begin{equation}
H^2 = 4\pi G\rho_m ,
\end{equation}
implying the exact identity
\begin{equation}\label{eq:chi_q_identity}
\chi_\delta = 1 \Longleftrightarrow q = 0 .
\end{equation}

The same structure appears at the generator level by rewriting the growth equation as a first-order system,
\begin{equation}\label{eq:Ldelta_def}
\frac{d}{dt}
\begin{pmatrix}
\delta \\ \dot{\delta}
\end{pmatrix}
=
\mathcal{L}_\delta
\begin{pmatrix}
\delta \\ \dot{\delta}
\end{pmatrix},
\qquad
\mathcal{L}_\delta =
\begin{pmatrix}
0 & 1 \\
4\pi G\rho_m & -2H
\end{pmatrix}.
\end{equation}
The eigenvalues of $\mathcal{L}_\delta$ are
\begin{equation}\label{eq:Ldelta_eigs}
\lambda_\pm = -H \pm \sqrt{H^2 - 4\pi G\rho_m},
\end{equation}
which coalesce precisely when $H^2 = 4\pi G\rho_m$, equivalent to $\chi_\delta=1$.

This parameter-free identity links cosmic acceleration onset to critical damping of structure growth.

\paragraph{Scope of the identity.}
Equation~\eqref{eq:chi_q_identity} holds \emph{exactly} for spatially flat $\Lambda$CDM. In non-flat cosmologies or in $w$CDM with $w\neq -1$, the synchronization between the acceleration transition and the critical-damping crossing is no longer exact, but the stability crossing $\chi_\delta=1$ remains well-defined and shifts to a nearby redshift. This makes departures from $\Lambda$CDM testable as measurable offsets between the $q=0$ and $\chi_\delta=1$ transitions. Figure~\ref{fig:synchronization} demonstrates this synchronization.

\begin{figure}[H]
\centering
\includegraphics[width=\linewidth]{Fig2_Synchronization.png}
\caption{\textbf{Cosmological synchronization of critical damping and acceleration onset.} \textbf{(a)} Temporal evolution showing the deceleration parameter $q(z)$ (blue) and stability index $\chi_\delta(z)$ (magenta dashed) as functions of redshift. The curves intersect at the redshift where both $q = 0$ and $\chi_\delta = 1$ simultaneously. \textbf{(b)} Phase space representation with matter density $\Omega_m$ and dark energy density $\Omega_\Lambda$ as coordinates. The cosmic trajectory is shown for the concordance values $\Omega_m \approx 0.3$ and $\Omega_\Lambda \approx 0.7$.}
\label{fig:synchronization}
\end{figure}

This result is parameter-free within flat $\Lambda$CDM and independent of any cosmological origin model \cite{planck2020,riess1998,perlmutter1999}.

% ============================================================
\section{Robustness Under Interactions and Memory}
% ============================================================

In weakly interacting $\lambda\phi^4$ theory, renormalization-group flow yields
\begin{equation}
\frac{d\chi}{d\ell} = (a_\gamma - a_\omega)\lambda \chi ,
\end{equation}
so that $\chi$ remains marginal when $a_\gamma \approx a_\omega$ \cite{wilson1974}. One-loop calculations show $|\Delta\chi| < 1\%$ over multiple decades \cite{christensen_qft}.

Finite-memory baths with kernel $K(t)=\gamma_0 e^{-t/\tau}$ yield an effective damping
\begin{equation}\label{eq:gamma_eff}
\gamma_{\mathrm{eff}}(\omega) = \frac{\gamma_0}{1+(\omega\tau)^2},
\end{equation}
broadening but not destroying the critical boundary \cite{breuer2002}. The physically realized damping ratio in such environments is therefore
\begin{equation}\label{eq:chi_phys}
\chi_{\mathrm{phys}}(\omega) \equiv \frac{\gamma_{\mathrm{eff}}(\omega)}{2\omega}
= Z(\omega\tau)\,\chi_{\mathrm{bare}},
\qquad
Z(\omega\tau)=\frac{1}{1+(\omega\tau)^2}<1,
\end{equation}
so that the exceptional point at $\chi_{\mathrm{bare}}=1$ corresponds to a downward-shifted operating point when expressed in terms of effective rates.

% ============================================================
\section{Substrate Inheritance as a Structural Relation}
% ============================================================

Let $\{\phi_k\}$ denote substrate modes with frequencies $\Omega_k$ and damping $\Gamma_k$. Any emergent mode $\psi = \sum_k c_k \phi_k$ inherits effective parameters
\begin{equation}
\Omega_\psi^2 = \sum_k |c_k|^2 \Omega_k^2, \qquad
\Gamma_\psi = \sum_k |c_k|^2 \Gamma_k .
\end{equation}
The resulting damping ratio is
\begin{equation}
\chi_\psi = \frac{\Gamma_\psi}{2\sqrt{\Omega_\psi^2}} .
\end{equation}

To make the overlap structure intrinsic, define the stabilized substrate subspace $\mathcal{H}_s$ and the corresponding orthogonal projector
\begin{equation}\label{eq:Ps_def}
\mathcal{P}_s \equiv \sum_{i\in s} |\phi_i\rangle\langle\phi_i|.
\end{equation}
For any proto-mode $|\psi_{\mathrm{proto}}\rangle$, the substrate overlap is defined geometrically as
\begin{equation}\label{eq:epsilon_def}
\epsilon_\psi \equiv \|\mathcal{P}_s |\psi_{\mathrm{proto}}\rangle\|^2
= \langle \psi_{\mathrm{proto}} | \mathcal{P}_s | \psi_{\mathrm{proto}} \rangle .
\end{equation}
In a substrate eigenbasis this reduces to $\epsilon_\psi=\sum_{i\in s}|\langle \phi_i|\psi_{\mathrm{proto}}\rangle|^2$.

Empirically, long-lived particles occupy the underdamped basin $\chi \ll 1$, while short-lived resonances cluster near $\chi \approx 1$. This distribution is shown in Figure~\ref{fig:lineage}.

\begin{figure}[H]
\centering
\includegraphics[width=0.95\linewidth]{Fig3_MechMatterLineage.png}
\caption{\textbf{Cross-scale empirical validation: stability classification across orders of magnitude.} Physical systems plotted as damping scale $\log_{10}\Gamma$ versus characteristic frequency $\log_{10}\omega$ from cosmological to Planck scales. \textbf{Red diamonds} indicate vacuum-lineage processes near $\chi = \Gamma/(2\omega) \approx 1$. \textbf{Gray arrows} illustrate schematic movement from near-critical regimes into the underdamped basin. For elementary particles, $\omega$ is identified with rest mass (natural units) and $\Gamma$ with the measured decay width; for effectively stable particles, $\Gamma$ represents an upper bound or interaction-limited width at the relevant scale \cite{pdg2022}. Particle data from PDG \cite{pdg2022}.}
\label{fig:lineage}
\end{figure}

\paragraph{Null-model comparator.}
Under a null model in which $(\omega,\Gamma)$ values are not constrained by a cross-scale organizing principle, the induced distribution of $\chi=\Gamma/(2\omega)$ would be broadly spread in log-space rather than compressed toward a narrow neighborhood. The observed clustering of vacuum-lineage processes near $\chi\approx 1$ and of long-lived matter at $\chi\ll 1$ is therefore not a generic consequence of plotting conventions, but a nontrivial empirical pattern consistent with stability-based classification.

When a dominant substrate satisfies $\chi \approx 1$, any emergent mode with nonzero overlap inherits a component of this near-critical structure. In the dominance limit, $|c_s|^2 \gg |c_{k\neq s}|^2$, the emergent ratio satisfies $\chi_\psi \rightarrow \chi_s$. Emergent parameters are therefore constrained projections of pre-existing substrate properties rather than independent tunings. The stability landscape is schematically represented in Figure~\ref{fig:potential}.

\begin{figure}[H]
\centering
\includegraphics[width=0.90\linewidth]{Fig4_StabilityBasin.png}
\caption{\textbf{Schematic stability landscape: state-space representation of the basin structure.} The revolved surface illustrates the stability hierarchy across the particle spectrum using an illustrative mapping in $\chi$-space. This functional form is a visualization aid, not a derived fundamental potential. The rim ($\chi = 1$) denotes the exceptional-point boundary separating overdamped and underdamped solution classes. Long-lived matter occupies the basin where detuning and impedance mismatch suppress rapid decay, while broad resonances cluster near the rim where partial resonance permits rapid decay.}
\label{fig:potential}
\end{figure}

\subsection{Structural Role of Gauge Couplings}
Within the stability architecture, gauge couplings are classified as \emph{substrate-ratio parameters}. In representative open-field settings, interaction-induced broadening rates admit scaling of the form $\Gamma_i \propto g_i^2\,\Omega_i$ up to model-dependent phase-space factors and kinematic thresholds, where $\Omega_i$ denotes a characteristic sector frequency at the scale of interest. The effective stability index $\chi_i = \Gamma_i/(2\Omega_i)$ therefore depends directly on the gauge coupling.

The persistence of a gauge sector as a coherent substrate requires operation within the near-critical window $\chi_i \approx 1$. This condition constrains $g_i^2$ to values that maintain spectral balance between interaction-induced damping and coherent propagation across renormalization-group flow. In this sense, gauge couplings are not freely tunable parameters, but ratios fixed by stability consistency.

The framework does not claim a parameter-free derivation of the numerical values of $(g_1,g_2,g_3)$. Instead, it provides a stability-based classification: the couplings control the ratio between interaction-induced broadening and coherent sector frequency, and therefore determine whether a sector admits near-critical operation in the sense of $\chi_i=\Gamma_i/(2\Omega_i)$. Any stronger claim requires specifying $(\Omega_i,\Gamma_i)$ operationally and demonstrating the mapping to the measured renormalized couplings at a fixed reference scale.

% ============================================================
\section{Experimental and Observational Tests}
% ============================================================

Having established the structural role of the $\chi=1$ boundary across quantum, field-theoretic, and cosmological dynamics, this section outlines concrete experimental and observational tests capable of confirming or falsifying the stability architecture.

The structural identity of the $\chi=1$ boundary yields falsifiable predictions across accessible energy scales. Each test below specifies an observable, a measurable threshold, and the conditions under which this framework would be challenged.

\subsection{Cosmological Synchronization (DESI \& Euclid)}
In flat $\Lambda$CDM, the onset of cosmic acceleration ($q=0$) and the critical damping of structure growth ($\chi_{\delta}=1$) occur at the same transition redshift $z_t$. Extensions such as curvature or dynamical dark energy ($w(z)\neq -1$) generically break this synchronization, producing an offset $\Delta z = z_{q=0} - z_{\chi=1}$.
\begin{itemize}
    \item \textbf{Observable:} The redshift difference $\Delta z$ between the kinematic transition ($q=0$) inferred from supernovae/BAO and the structural transition ($\chi_{\delta}=1$) inferred from growth-rate measurements ($f\sigma_8$).
    \item \textbf{Sensitivity Requirement:} Upcoming surveys (DESI, Euclid, Rubin) are expected to constrain $z_t \approx 0.6$ with precision $\sigma_z < 0.05$. A statistically significant offset $|\Delta z| > 3\sigma_z$ that cannot be attributed to curvature or $w(z)$ evolution would challenge the structural identity.
\end{itemize}

\subsection{Quantum Spectral Merger (Circuit QED)}
The transition from underdamped oscillation to overdamped decay at $\chi=1$ predicts a unique spectral signature: the coalescence of eigenfrequencies into a second-order exceptional point (EP2). This can be probed in superconducting qubit–cavity systems with tunable coupling $g$.
\begin{itemize}
    \item \textbf{Protocol:} A transmon qubit ($\omega_q/2\pi \approx 5$ GHz) coupled to a dissipative cavity ($\kappa/2\pi \approx 1$ MHz) is tuned through the exceptional point by varying the Purcell decay rate $\gamma_P = \kappa g^2/\Delta^2$.
    \item \textbf{Signature:} As $\chi$ is scanned through $\{0.8,0.9,1.0,1.1\}$, the qubit population $P_1(t)$ must transition from oscillatory decay $e^{-\gamma t/2}\cos(\omega_a t)$ to the EP2 kernel $t e^{-\omega t}$ at the critical point, accompanied by spectral doublet merger $\omega_{\pm}\to\omega_0$.
    \item \textbf{Sensitivity Requirement:} Time-domain sampling with $\delta t \le 10$ ns over a $10\,\mu$s window is required to resolve the critical power-law tail. A failure to observe spectral coalescence within the expected window $0.95 < \chi < 1.05$ would challenge the existence of the EP2 boundary in open quantum dynamics.
\end{itemize}

\subsection{Finite-Temperature QCD Coherence (Lattice)}
If the QCD substrate participates in near-critical stability organization, lattice determinations of finite-temperature spectral functions should reveal a scalar ($0^{++}$) channel with damping ratio $\chi \equiv \Gamma/(2\Omega)$ approaching unity in the vicinity of the confinement transition.
\begin{itemize}
    \item \textbf{Observable:} The thermal pole or peak location $\Omega(T)$ and width $\Gamma(T)$ extracted from spectral reconstructions in the $0^{++}$ channel.
    \item \textbf{Signature:} Existence of a temperature band near the transition where $0.8 \lesssim \chi(T) \lesssim 1.0$.
    \item \textbf{Challenge condition:} Absence of any near-critical window in all plausible reconstruction methods would weaken the substrate-coherence interpretation.
\end{itemize}

\subsection{Neutron Star Ringdown (Gravitational Waves)}
For neutron-star $f$-modes, the damping time $\tau$ and oscillation frequency $f$ define a dimensionless stability index $\chi_{\mathrm{NS}} = (2\pi f\tau)^{-1}$. As compact objects approach the TOV limit, the mode trajectory is predicted to approach the structural boundary $\chi_{\mathrm{NS}} \approx 1$.
\begin{itemize}
    \item \textbf{Test:} Third-generation gravitational-wave detectors (Cosmic Explorer, Einstein Telescope) should observe that remnant objects populate the region $\chi_{\mathrm{NS}} \le 1$; a statistically significant population with $\chi_{\mathrm{NS}} > 1$ would contradict the stability-boundary interpretation.
\end{itemize}

% ============================================================
\section{Conclusion}
% ============================================================

A single dimensionless ratio, $\chi=\gamma/(2|\omega|)$, defines a stability architecture across quantum, field-theoretic, and cosmological dynamics. The critical boundary $\chi=1$ marks a non-Hermitian exceptional point that coincides exactly with the onset of cosmic acceleration via the identity $\chi_\delta = 1 \Longleftrightarrow q = 0$.
Substrate inheritance constrains emergent parameters without invoking phenomenological tuning, and observed particle hierarchies are consistent with stability-based classification.

\paragraph{Interpretation.}
Any apparent preference for $\chi\approx 1$ is interpreted as survivorship under stability constraints rather than as an optimization principle imposed by nature. In this view, the critical boundary functions as a separatrix that organizes which structures can persist, not as a target that dynamics must seek. These results establish $\chi$ as a structural organizing principle of physical reality.

\paragraph{Falsifiability.}
The framework yields several independent tests.
(i) In flat $\Lambda$CDM, the transitions $\chi_\delta=1$ and $q=0$ must coincide; any observed offset would falsify the structural identity.
(ii) Finite-temperature lattice QCD must exhibit a $0^{++}$ mode with $\chi\approx 1$ near the confinement transition.
(iii) Quantum platforms with tunable $\gamma/\omega$ must display spectral-peak merger and the EP2 kernel at $\chi=1$.
(iv) Neutron-star oscillation modes should approach $\chi_{\mathrm{NS}}\to 1$ near the TOV limit.
These tests collectively determine whether the stability architecture described here is realized in nature.

\begin{figure}[H]
\centering
\includegraphics[width=0.95\linewidth]{Fig5_StabilityGeometry.png}
\caption{\textbf{Unified topology: vacuum-to-spacetime structure.} The funnel represents the stability manifold in $\chi$-space, separating overdamped ($\chi > 1$), exceptional ($\chi = 1$), and underdamped ($\chi < 1$) dynamical regimes. This is parameter space, not spatial geometry. The visualization provides a stability-theoretic correspondence with general-relativistic curvature and does not replace Einstein dynamics.}
\label{fig:geometry}
\end{figure}

\begin{thebibliography}{99}

\bibitem{abbott2020}
Abbott, B. P., et al. (2020). GW190814: Gravitational waves from coalescence. \textit{Astrophysical Journal Letters}, 896, L44.

\bibitem{aspelmeyer2014}
Aspelmeyer, M., Kippenberg, T. J., \& Marquardt, F. (2014). Cavity optomechanics. \textit{Reviews of Modern Physics}, 86, 1391-1452.

\bibitem{bender2007}
Bender, C. M. (2007). Making sense of non-Hermitian Hamiltonians. \textit{Reports on Progress in Physics}, 70, 947-1018.

\bibitem{blais2021}
Blais, A., et al. (2021). Circuit quantum electrodynamics. \textit{Reviews of Modern Physics}, 93, 025005.

\bibitem{breuer2002}
Breuer, H.-P., \& Petruccione, F. (2002). \textit{The Theory of Open Quantum Systems}. Oxford University Press.

\bibitem{cardoso2016}
Cardoso, V., Franzin, E., \& Pani, P. (2016). Is the gravitational-wave ringdown a probe of the event horizon? \textit{Physical Review Letters}, 116, 171101.

\bibitem{carmichael1999}
Carmichael, H. J. (1999). \textit{Statistical Methods in Quantum Optics 1}. Springer.

\bibitem{clerk2010}
Clerk, A. A., et al. (2010). Introduction to quantum noise, measurement, and amplification. \textit{Reviews of Modern Physics}, 82, 1155-1208.

\bibitem{desi2024}
DESI Collaboration. (2024). DESI 2024 VI: Cosmological constraints from BAO. arXiv:2404.03002.

\bibitem{dodelson2003}
Dodelson, S. (2003). \textit{Modern Cosmology}. Academic Press.

\bibitem{gorini1976}
Gorini, V., Kossakowski, A., \& Sudarshan, E. C. G. (1976). Completely positive dynamical semigroups of N-level systems. \textit{Journal of Mathematical Physics}, 17, 821-825.

\bibitem{heiss2012}
Heiss, W. D. (2012). The physics of exceptional points. \textit{Journal of Physics A: Mathematical and Theoretical}, 45, 444016.

\bibitem{kato1995}
Kato, T. (1995). \textit{Perturbation Theory for Linear Operators}. Springer.

\bibitem{lindblad1976}
Lindblad, G. (1976). On the generators of quantum dynamical semigroups. \textit{Communications in Mathematical Physics}, 48, 119-130.

\bibitem{ogata2010}
Ogata, K. (2010). \textit{Modern Control Engineering} (5th ed.). Prentice Hall.

\bibitem{pdg2022}
Particle Data Group. (2022). Review of particle physics. \textit{Progress of Theoretical and Experimental Physics}, 2022, 083C01.

\bibitem{peebles1993}
Peebles, P. J. E. (1993). \textit{Principles of Physical Cosmology}. Princeton University Press.

\bibitem{perlmutter1999}
Perlmutter, S., et al. (1999). Measurements of $\Omega$ and $\Lambda$ from 42 high-redshift supernovae. \textit{Astrophysical Journal}, 517, 565-586.

\bibitem{planck2020}
Planck Collaboration. (2020). Planck 2018 results. VI. Cosmological parameters. \textit{Astronomy \& Astrophysics}, 641, A6.

\bibitem{riess1998}
Riess, A. G., et al. (1998). Observational evidence from supernovae for an accelerating universe. \textit{Astronomical Journal}, 116, 1009-1038.

\bibitem{wilson1974}
Wilson, K. G., \& Kogut, J. (1974). The renormalization group and the $\epsilon$ expansion. \textit{Physics Reports}, 12, 75-199.

\bibitem{christensen_primordial}
Christensen, N. (2026). The Primordial Boundary Principle: Identifying Cosmic Acceleration with Exceptional Point Coalescence. \textit{Zenodo}. \url{https://doi.org/10.5281/zenodo.17490497}

\bibitem{christensen_cosmo}
Christensen, N. (2026). Structural Mapping of Linear Damping Operators Across Cosmological Growth and Black Hole Ringdown. \textit{Zenodo}. \url{https://doi.org/10.5281/zenodo.17503537}

\bibitem{christensen_qft}
Christensen, N. (2026). Structural Constraints from Critical Damping in Open Quantum Field Theories: Implications for QCD Substrate Inheritance and Phenomenological Extensions. \textit{Zenodo}. \url{https://doi.org/10.5281/zenodo.17437688}

\bibitem{christensen_neutrinos}
Christensen, N. (2026). Density-Dependent Matter-Induced Dephasing in Neutrino Oscillations with Preserved Vacuum Unitarity
. \textit{Zenodo}. \url{https://doi.org/10.5281/zenodo.17585527}

\bibitem{christensen_gaps}
Christensen, N. (2026). Closing Critical Gaps: Physical Inheritance from Stabilized Substrates in Dynamical Systems. \textit{Zenodo}. \url{https://doi.org/10.5281/zenodo.17428940}

\end{thebibliography}

\end{document}